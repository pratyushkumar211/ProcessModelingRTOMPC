\documentclass[xcolor=dvipsnames, 8pt]{beamer} %
%\setbeamertemplate{navigation symbols}{}

\usetheme{SantaBarbara}

\definecolor{black}{HTML}{0A0A0A}
\definecolor{green}{HTML}{0420eb} 
\definecolor{darkgreen}{HTML}{008000}
\definecolor{gold}{HTML}{FFD000}
\setbeamercolor{normal text}{fg=black,bg=white}
\setbeamercolor{alerted text}{fg=red}
\setbeamercolor{example text}{fg=black}
\setbeamercolor{palette primary}{fg=black, bg=gray!20}
\setbeamercolor{palette secondary}{fg=black, bg=gray!20}
\setbeamercolor{palette tertiary}{fg=white, bg=green!80}
\setbeamercolor{block title}{fg=black,bg=gold!40}
\setbeamercolor{frametitle}{fg=white, bg=green!80}
\setbeamercolor{title}{fg=white, bg=green!80}

\newcommand{\vecc}{\operatorname{vec}}
%\newcommand{\svec}{\operatorname{svec}}
\newcommand{\norm}[1]{\left| #1 \right|}
\newcommand{\xhat}{\hat{x}}
\newcommand{\uhat}{\hat{u}}
\newcommand{\yhat}{\hat{y}}
\newcommand{\wt}{\widehat{\theta}}

\usepackage[utf8]{inputenc}

\usepackage{tikz, tikzsettings}
\usepackage{verbatim}
\usepackage{amssymb}
\usepackage{amsmath}
%\usepackage{amsfonts}
\usepackage{algorithmic}
\graphicspath{{./}{./figures/}{./figures/presentation/}}
%\usepackage[version=4]{mhchem}
\usepackage{subcaption}
\usepackage[authoryear,round]{natbib}
%\usepackage{fancyvrb}
\usepackage{color}
\usepackage{colortbl}
\usepackage{xcolor}
\usepackage{physics}
\usepackage{pgfplots}
\usepackage{ragged2e}
%\pgfplotsset{compat=newest} 
%\pgfplotsset{plot coordinates/math parser=false}
%\usepackage{environ}
%\usetikzlibrary{decorations.markings}
%\usetikzlibrary{decorations.pathreplacing}
\usetikzlibrary{shapes,calc,spy, calc, backgrounds,arrows, fit, decorations.pathmorphing, decorations.pathreplacing, matrix}
%\usepackage[absolute,overlay]{textpos}
\usepackage{caption}
\usepackage{graphicx}
%\usepackage[controls=true,poster=first]{animate}

\newcommand{\useq}{\mathbf{u}}
\newcommand{\xseq}{\mathbf{x}}
\newcommand{\bbR}{\mathbb{R}}
\newcommand{\bbW}{\mathbb{W}}
\newcommand{\bbU}{\mathbb{U}}
\newcommand{\bbI}{\mathbb{I}}
\newcommand{\bbX}{\mathbb{X}}
\newcommand{\bbP}{\mathbb{P}}


\AtBeginSection[]
{\frame<beamer>{\frametitle{Outline}   
	\tableofcontents[currentsection, currentsection]}
	\addtocounter{framenumber}{-1}}
	%
%{
%	\frame<beamer>{ 
%		\frametitle{Outline}   
%		\tableofcontents[currentsection,currentsubsection]}
		
%}

\makeatother
\setbeamertemplate{footline}
{
	\leavevmode%
	\hbox{%
		\begin{beamercolorbox}[wd=.3\paperwidth,ht=2.25ex,dp=1ex,center]{author in head/foot}%
			\usebeamerfont{author in head/foot}\insertshortauthor
		\end{beamercolorbox}%
		\begin{beamercolorbox}[wd=.6\paperwidth,ht=2.25ex,dp=1ex,center]{title in head/foot}%
			\usebeamerfont{title in head/foot}\insertshorttitle
		\end{beamercolorbox}%
		\begin{beamercolorbox}[wd=.1\paperwidth,ht=2.25ex,dp=1ex,center]{date in head/foot}%
			\insertframenumber{} / \inserttotalframenumber\hspace*{1ex}
	\end{beamercolorbox}}%
	\vskip0pt%
}


\title{Hybrid system identification using neural networks}
\date{Jan 19, 2021}
\author[Pratyush Kumar]{\large Pratyush Kumar}
\institute[UCSB]{
	\begin{minipage}{4in}
		\vspace{-10pt}
		\centering
		\raisebox{-0.1\height}{\includegraphics[width=0.25\textwidth]{UCSB_seal}}
		%\hspace*{.2in}
		%\raisebox{-0.5\height}{\includegraphics[width=0.25\textwidth]{jci_logo}}
	\end{minipage}
	\vspace{10pt}
	\newline
	{\large Department of chemical engineering}
	\vspace{10pt}
	\newline
	{\large Group talk}}

\begin{document}

\frame{\titlepage}
	
\begin{frame}{Outline}
\tableofcontents
\end{frame}

\section{Introduction}

\begin{frame}{Plant and grey-box model}

	\begin{block}{Plant}
		\begin{align*}
			\dot x_p &= f_p(x_p, u, w) \\
			y &= h_p(x_p) + v
		  \end{align*}
	$x_p \in \bbR^{n_p}$, $y \in \bbR^{n_y}$, and $u \in \bbU$.
	\end{block}
		
	\begin{block}{Grey-box model}
		\begin{align*}
			\dot x_g &= f_g(x_g, u) \\
			y &= h_g(x_g)
		  \end{align*}
	$x_g \in \bbR^{n_g}$, usually $n_g \leq n_p$.
	\end{block}
	A crude grey-box model is often adequate for setpoint tracking MPC applications, but performance can be improved in dynamic 
	economic optimization problems using better and accurate models. 
	\end{frame}
	
\begin{frame}{Modeling choices}
	Define - 
	\begin{align*}
	\mathbf{y}_{k-N_p:k-1} &= [y(k-\Delta)', y(k-2\Delta)', ..., y(k-N_p\Delta)']' \\
	\mathbf{u}_{k-N_p:k-1} &= [u(k-\Delta)', u(k-2\Delta)', 
										 ..., u(k-N_p\Delta)']' \\
	z(k) &= [\mathbf{y}'_{k-N_p:k-1}, \mathbf{u}'_{k-N_p:k-1}]'
	\end{align*}	
	in which, $N_p$ is the number of past inputs and outputs used. 
	\begin{block}{Black-box}
	$y^+ = f_N(y, z, u)$
	\end{block}
	
	\begin{block}{Hybrid: Grey-box and neural network}
		\begin{align*}
			\dot x_g &= f_g(x_g, u) + f_N(x_g, z, u)\\
			y &= h_g(x_g)
		\end{align*}
	$f_N(\cdot)$ is a neural network.
	\end{block}
			
\end{frame}

\section{Reactor example}

\begin{frame}{Nonlinear reactor model}

\textbf{Plant}
\begin{align*}
	\dfrac{dC_A}{dt} &= \dfrac{C_{Af} - C_A}{\tau} - k_1C_A\\
	\dfrac{dC_B}{dt} &= k_1C_A - 3k_2C^3_B + 3k_3C_C- \dfrac{C_B}{\tau}\\
	\dfrac{dC_C}{dt} &= k_2C^3_B - k_3C_C - \dfrac{C_C}{\tau}
\end{align*}
$u = C_{Af}$, $x = [C_A, C_B, C_C]'$, $y = [C_A, C_B]'$

\vspace{0.2in}
\textbf{Grey-box model}
\begin{align*}
  \dfrac{dC_A}{dt} &= \dfrac{C_{Af} - C_A}{\tau} - k_1C_A\\
  \dfrac{dC_B}{dt} &= k_1C_A - \dfrac{C_B}{\tau}
\end{align*}

\vspace{0.2in}
Train black-box and hybrid models with NN 
architectures of $[9, \ 16, \ 2]$ and $[8, \ 16, \ 2]$
on 480 samples (8 hours).
\end{frame}

\begin{frame}{Model validation -- open loop data}
\vspace{-0.1in}
\begin{figure}
\centering
\includegraphics[page=1, height=0.9\textheight, 
				 width=0.7\textwidth]{tworeac_plots_nonlin.pdf}
\end{figure}
\vspace{-0.2in}
Pretty good multi-step ahead predictions on validation data 
by both the models. 
\end{frame}

\begin{frame}{Economic MPC problem}
Stage cost.
\begin{align*}
		\ell(y, u) = c_aC_{Af} - c_bC_B
\end{align*}
Economic MPC problem.
\begin{align*}
	\min_{\useq} & 
  \sum_{k=0}^{N-1} \ell(y, u, p) \\
	\textnormal{s.t.} \quad   x^+ &= f(x, u)\\
	y &= h(x) \\ 
	u &\in \bbU
  \end{align*}
  in which $p$ are the time-varying parameters $c_a$ and $c_b$. \\
  Choose a horizon length of $N=60$ (1 hour). 
\end{frame}
	
\begin{frame}{Economic MPC parameters}
	\begin{figure}
	\centering
	\includegraphics[page=4, height=0.9\textheight, 
					 width=0.7\textwidth]{tworeac_plots_nonlin.pdf}
	\end{figure}
\end{frame}

\begin{frame}{Open-loop economic MPC solution}
	\vspace{-0.1in}
	\begin{figure}
	\centering
	\includegraphics[page=2, height=0.9\textheight, 
					 width=0.7\textwidth]{tworeac_plots_nonlin.pdf}
	\end{figure}
	\vspace{-0.2in}
	IPOPT gives unreliable solution for the hybrid model.
\end{frame}

\begin{frame}{Closed-loop simulation}
	\begin{figure}
	\centering
	\includegraphics[page=3, height=0.9\textheight, 
					 width=0.7\textwidth]{tworeac_plots_nonlin.pdf}
	\end{figure}
\end{frame}

\begin{frame}{Closed-loop simulation - Average Stage Costs}
	\centering
	$\Lambda_k = \dfrac{1}{k}\sum_{i=0}^{k-1} \ell(y(k), u(k), p(k))$
	\vspace{-0.05in}
	\begin{figure}
	\includegraphics[page=5, height=0.85\textheight, 
					 width=0.65\textwidth]{tworeac_plots_nonlin.pdf}
	\end{figure}
\end{frame}

\section{CSTR and Flash example}

\begin{frame}{CSTR and Flash example}
\begin{figure}
	\includegraphics[height=0.5\textheight, 
					 width=0.6\textwidth]{cstr_flash.pdf}
\end{figure}
\begin{itemize}
	\item Two exothermic reactions $A \rightarrow B$ and $3B \rightarrow C$
	in the CSTR.
	\item 10 states, 4 manipulated inputs, and 6 measurements.
\end{itemize}
\end{frame}

\begin{frame}{Plant and grey-box model}
	\textbf{CSTR balances}
	\vspace{0.2in}
	\begin{columns}
		\begin{column}{0.45\textwidth}
			\textbf{Plant} \\
			\begin{align*}				
			\dfrac{dH_r}{dt} &= \dfrac{F + D -F_r}{A_r}\\
			\dfrac{dC_{Ar}}{dt} &= \dfrac{F(C_{Af} -C_{Ar}) +
								   D(C_{Ad} -C_{Ar})}{A_rH_r} - r_1 \\
			\dfrac{dC_{Br}}{dt} &= \dfrac{-FC_{Br} + 
									D(C_{Bd} -C_{Br})}{A_rH_r} + r_1 -3r_2\\
			\dfrac{dC_{Cr}}{dt} &= \dfrac{-FC_{Cr} + 
			D(C_{Cd} -C_{Cr})}{A_rH_r} + r_2\\
			\dfrac{dT_r}{dt} &= \dfrac{F(T_f - T_r) + D(T_d -T_r)}{A_rH_r} \\
						& + \dfrac{r_1\Delta H_1 + r_2\Delta H_2}{\rho C_p} + 
							\dfrac{Q_r}{\rho A_r C_p H_r}\\
			\end{align*}
		\end{column}
		\begin{column}{0.45\textwidth}
			\textbf{Grey-box Model}
			\begin{align*}
			\dfrac{dH_r}{dt} &= \dfrac{F + D -F_r}{A_r}\\
			\dfrac{dC_{Ar}}{dt} &= \dfrac{F(C_{Af} -C_{Ar}) + 
								   D(C_{Ad} -C_{Ar})}{A_rH_r} - r_1 \\
			\dfrac{dC_{Br}}{dt} &= \dfrac{-FC_{Br} + 
									D(C_{Bd} -C_{Br})}{A_rH_r} + r_1\\
			\dfrac{dT_r}{dt} &= \dfrac{F(T_f - T_r) + D(T_d -T_r)}{A_rH_r} \\ 
							& + \dfrac{r_1\Delta H_1}{\rho C_p} + 
								\dfrac{Q_r}{\rho A_r C_p H_r}\\
			\end{align*}
		\end{column}
	\end{columns}
The second reaction $3B \rightarrow C$ is ignored in the grey-box model.
\end{frame}

\begin{frame}{Plant and grey-box model}
	\textbf{Flash balances}
	\vspace{0.2in}
	\begin{columns}
		\begin{column}{0.45\textwidth}
			\textbf{Plant} \\
			\begin{align*}
				\dfrac{dH_b}{dt} &= \dfrac{F_r - F_b - D}{A_b} \\
				\dfrac{dC_{Ab}}{dt} &= \dfrac{F_r(C_{Ar} -C_{Ab}) + 
										D(C_{Ab} -C_{Ad})}{A_bH_b} \\
				\dfrac{dC_{Bb}}{dt} &= \dfrac{F_r(C_{Br} -C_{Bb}) + 
										D(C_{Bb} -C_{Bd})}{A_bH_b} \\
				\dfrac{dC_{Cb}}{dt} &= \dfrac{F_r(C_{Cr} -C_{Cb}) + 
										D(C_{Cb} -C_{Cd})}{A_bH_b} \\
				\dfrac{dT_b}{dt} &= \dfrac{F_r(T_r - T_b)}{A_bH_b} + \dfrac{Q_b}{\rho A_b C_p H_b}\\
			  \end{align*}		
		\end{column}
		\begin{column}{0.45\textwidth}
			\textbf{Grey-box Model}
			\begin{align*}
				\dfrac{dH_b}{dt} &= \dfrac{F_r - F_b - D}{A_b} \\
				\dfrac{dC_{Ab}}{dt} &= \dfrac{F_r(C_{Ar} -C_{Ab}) + 
										D(C_{Ab} -C_{Ad})}{A_bH_b} \\
				\dfrac{dC_{Bb}}{dt} &= \dfrac{F_r(C_{Br} -C_{Bb}) + 
										D(C_{Bb} -C_{Bd})}{A_bH_b} \\
				\dfrac{dT_b}{dt} &= \dfrac{F_r(T_r - T_b)}{A_bH_b} +
									\dfrac{Q_b}{\rho A_b C_p H_b}\\
			\end{align*}
		\end{column}
	\end{columns}
	Six measurements: $H_r, C_{Ar}, T_r, H_b, C_{Ab}, T_b$. \\
	Train black-box and hybrid models with NN 
	architectures of $[60, \ 32, \ 6]$ and $[62, \ 32, \ 8]$
	on 3240 samples (54 hours).	
\end{frame}

\begin{frame}{Model validation -- Input Sequence}
	\begin{figure}
	\centering
	\includegraphics[page=1, height=0.9\textheight, 
					 width=0.7\textwidth]{cstr_flash_plots.pdf}
	\end{figure}
\end{frame}
	
\begin{frame}{Model validation -- Outputs}
	\vspace{-0.1in}
	\begin{figure}
	\centering
	\includegraphics[page=2, height=0.85\textheight, 
					 width=0.65\textwidth]{cstr_flash_plots.pdf}
	\end{figure}
	\vspace{-0.2in}
Pretty good multi-step ahead predictions on validation data 
by the hybrid model, black-box model is also okay. 
\end{frame}

\begin{frame}{Economic MPC problem}
	Stage cost.
\begin{align*}
\ell(y, u) = c_aFC_{Af} + c_eQ_r + 
		c_eQ_b + c_eD\rho C_p(T_d-T_b) - 
		c_bF_bC_{Bb} \\
\end{align*}
Economic MPC problem.
\begin{align*}
	\min_{\useq} & 
  \sum_{k=0}^{N-1} \ell(y, u, p) \\
	\textnormal{s.t.} \quad   x^+ &= f(x, u)\\
	y &= h(x) \\ 
	u &\in \bbU
  \end{align*}
  in which $p$ are the time-varying parameters $c_e$, $c_a$, and $c_b$. \\
  Choose a horizon length of $N=60$ (1 hour). 
\end{frame}
		
\begin{frame}{Economic MPC parameters}
		\begin{figure}
		\centering
		\includegraphics[page=9, height=0.9\textheight, 
						 width=0.7\textwidth]{cstr_flash_plots.pdf}
		\end{figure}
\end{frame}
	
\begin{frame}{Open-loop economic MPC solution -- Inputs}
	\vspace{-0.1in}
		\begin{figure}
		\centering
		\includegraphics[page=4, height=0.9\textheight, 
						 width=0.7\textwidth]{cstr_flash_plots.pdf}
		\end{figure}
		\vspace{-0.2in}
	IPOPT gives unreliable solution for the hybrid model.
\end{frame}
	
\begin{frame}{Open-loop economic MPC solution -- States}
	\vspace{-0.1in}
	\begin{figure}
	\centering
	\includegraphics[page=5, height=0.9\textheight, 
					 width=0.7\textwidth]{cstr_flash_plots.pdf}
	\end{figure}
	\vspace{-0.2in}
	IPOPT gives unreliable solution for the hybrid model.
\end{frame}

\begin{frame}{Closed-loop simulation -- Inputs}
		\begin{figure}
		\centering
		\includegraphics[page=6, height=0.9\textheight, 
						 width=0.7\textwidth]{cstr_flash_plots.pdf}
		\end{figure}
\end{frame}

\begin{frame}{Closed-loop simulation -- Outputs}
	\begin{figure}
	\centering
	\includegraphics[page=7, height=0.9\textheight, 
					 width=0.7\textwidth]{cstr_flash_plots.pdf}
	\end{figure}
\end{frame}

\begin{frame}{Closed-loop simulation -- States}
	\begin{figure}
	\centering
	\includegraphics[page=8, height=0.9\textheight, 
					 width=0.7\textwidth]{cstr_flash_plots.pdf}
	\end{figure}
\end{frame}

\begin{frame}{Closed-loop simulation - Average Stage Costs}
	\centering
	$\Lambda_k = \dfrac{1}{k}\sum_{i=0}^{k-1} \ell(y(k), u(k), p(k))$
	\vspace{-0.05in}
		\begin{figure}
		\centering
		\includegraphics[page=10, height=0.85\textheight, 
						 width=0.65\textwidth]{cstr_flash_plots.pdf}
		\end{figure}
\end{frame}

\section{Conclusion}
\begin{frame}{Next steps.}
\begin{itemize}
	\item  Koopman Operators.
\end{itemize}  
\end{frame}

%\begin{frame}{References}
%\bibliographystyle{abbrvnat}
%\bibliography{articles,proceedings,books,unpub}
%\end{frame}

\end{document}
