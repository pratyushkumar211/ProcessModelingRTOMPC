\documentclass[xcolor=dvipsnames, 8pt]{beamer} %
%\setbeamertemplate{navigation symbols}{}

\usetheme{SantaBarbara}

\definecolor{black}{HTML}{0A0A0A}
\definecolor{green}{HTML}{0420eb} 
\definecolor{darkgreen}{HTML}{008000}
\definecolor{gold}{HTML}{FFD000}
\setbeamercolor{normal text}{fg=black,bg=white}
\setbeamercolor{alerted text}{fg=red}
\setbeamercolor{example text}{fg=black}
\setbeamercolor{palette primary}{fg=black, bg=gray!20}
\setbeamercolor{palette secondary}{fg=black, bg=gray!20}
\setbeamercolor{palette tertiary}{fg=white, bg=green!80}
\setbeamercolor{block title}{fg=black,bg=gold!40}
\setbeamercolor{frametitle}{fg=white, bg=green!80}
\setbeamercolor{title}{fg=white, bg=green!80}

\newcommand{\svec}{\operatorname{svec}}
\usepackage[version=4]{mhchem}

\usepackage[utf8]{inputenc}

\usepackage{tikz, tikzsettings}
\usepackage{verbatim}
\usepackage{amssymb}
\usepackage{amsmath}
%\usepackage{amsfonts}
\usepackage{algorithmic}
\graphicspath{{./}{./figures/}{./figures/presentation/}}
%\usepackage[version=4]{mhchem}
\usepackage{subcaption}
\usepackage[authoryear,round]{natbib}
%\usepackage{fancyvrb}
\usepackage{color}
\usepackage{colortbl}
\usepackage{xcolor}
\usepackage{physics}
\usepackage{pgfplots}
\usepackage{ragged2e}
%\pgfplotsset{compat=newest} 
%\pgfplotsset{plot coordinates/math parser=false}
%\usepackage{environ}
%\usetikzlibrary{decorations.markings}
%\usetikzlibrary{decorations.pathreplacing}
\usetikzlibrary{shapes,calc,spy, calc, backgrounds,arrows, fit, decorations.pathmorphing, decorations.pathreplacing, matrix}
%\usepackage[absolute,overlay]{textpos}
\usepackage{caption}
\usepackage{graphicx}
%\usepackage[controls=true,poster=first]{animate}


\AtBeginSection[]
{\frame<beamer>{\frametitle{Outline}   
	\tableofcontents[currentsection, currentsection]}
	\addtocounter{framenumber}{-1}}
	%
%{
%	\frame<beamer>{ 
%		\frametitle{Outline}   
%		\tableofcontents[currentsection,currentsubsection]}
		
%}

\makeatother
\setbeamertemplate{footline}
{
	\leavevmode%
	\hbox{%
		\begin{beamercolorbox}[wd=.3\paperwidth,ht=2.25ex,dp=1ex,center]{author in head/foot}%
			\usebeamerfont{author in head/foot}\insertshortauthor
		\end{beamercolorbox}%
		\begin{beamercolorbox}[wd=.6\paperwidth,ht=2.25ex,dp=1ex,center]{title in head/foot}%
			\usebeamerfont{title in head/foot}\insertshorttitle
		\end{beamercolorbox}%
		\begin{beamercolorbox}[wd=.1\paperwidth,ht=2.25ex,dp=1ex,center]{date in head/foot}%
			\insertframenumber{} / \inserttotalframenumber\hspace*{1ex}
	\end{beamercolorbox}}%
	\vskip0pt%
}


\title[Nonlinear system identification with neural networks for 
process control]{Nonlinear system identification with neural networks for 
process control}
\date{September 15, 2020}
\author[Kumar and Rawlings]{\large Pratyush Kumar and James B. Rawlings}
\institute[UCSB]{
	\begin{minipage}{4in}
		\vspace{-10pt}
		\centering
		\raisebox{-0.1\height}{\includegraphics[width=0.25\textwidth]{UCSB_seal}}
		%\hspace*{.2in}
		%\raisebox{-0.5\height}{\includegraphics[width=0.25\textwidth]{jci_logo}}
	\end{minipage}
	\vspace{10pt}
	\newline
	{\large Department of Chemical Engineering}
	\vspace{10pt}
	\newline
	{\large TWCCC Meeting (Virtual)}}
	%\vspace{10pt}
	%\newline
	%{\large Austin, TX}}

\begin{document}

\frame{\titlepage}

% MPC intro frame
\begin{frame}{System identification for model predictive control (MPC)}

MPC requires a dynamic plant model ($x^+ = f(x, u)$) for use in online optimization.  

	\begin{block}{Standard industrial practice}
		\begin{itemize}
			\item Estimate a linear model from data and use integrating disturbance models to maintain zero offset 
			in controlled measurements.
			\item For highly nonlinear plants, the closed-loop controller performance can deteriorate due to significant model mismatch. 
		\end{itemize}
	\end{block}
	
	\begin{block}{Neural networks and system identification\footnote[frame]{\cite{hussain:1999, draeger:engell:ranke:1995, lovelett:avalos:kevrekidis:2019, tuor:drgona:vrabie:2020}}}
		\begin{itemize}
			\item \textbf{Black-box identification:} Estimate the entire predictive model ($x^+ =  f(x, u)$) from actuators to
			sensors using data collected from open-loop identification experiments.
			\item \textbf{Hybrid model identification:} Model the dynamics from actuators to sensors using first-principle knowledge, and use
			neural networks (NNs) to estimate the mismatch or any unmeasured disturbance dynamics.  
		\end{itemize}
	\end{block}
\end{frame}

\begin{frame}{Nonlinear HVAC model}
	
	\centering
	\colorlet{CoolFlow}{NavyBlue!60!Cyan}
	\colorlet{OtherFlow}{black}
	\colorlet{HeatFlow}{Red}
	\colorlet{BuildingColor}{Brown}
	\colorlet{HVACColor}{Cerulean}
	\colorlet{SunColor}{yellow}
	
	\tikzset{>=stealth'}
	\resizebox{.6\textwidth}{!}{
	\begin{tikzpicture}
	[
	scale=1,
	flow/.style={->, thick, line cap=rect},
	flowlabel/.style={black, align=center, font=\scriptsize\itshape},
	masszonestyle/.style={draw,rectangle,minimum width=2.5cm,minimum height=3cm,color=black,ultra thick,fill=BuildingColor,fill opacity=0.6,text=black,text opacity=1},
	airzonestyle/.style={draw,rectangle,minimum width=1.5cm,minimum height=2cm,color=black,ultra thick,fill=BuildingColor!5,fill opacity=0.8,text=black,text opacity=1},
	hvacstyle/.style={draw,rectangle,minimum width=2cm,minimum height=1.5cm,color=black,ultra thick,fill=HVACColor,fill opacity=0.65,text=black,text opacity=1},
	]
	% Grid for drawing.
	% \draw (-5,-3) to[grid with coordinates] (5,3);
	
	% Draw the sun.
	\draw (-4,2) node[line width=1pt,draw=black,fill=SunColor,circle,minimum size=1cm] (TheSun) {};
	\foreach \angle in {0,45,...,315}
	{\draw [rotate around={\angle:(TheSun.center)}] ($(TheSun.center) + (0.7,0)$) node[shape border rotate=\angle-90,line width=1pt,draw=black,fill=SunColor,regular polygon,regular polygon sides=3,inner sep=0.05cm] {};}
	
	% Building.
	\node (building_rect) at (-1.25,-1.5) [masszonestyle] {};
	\node (building_rect) at (1.25,-1.5) [masszonestyle] {};
	\node (building_rect) at (-1.25,-1.5) [airzonestyle,label={[label distance=-1cm]90:Zone 1},label={[label distance =-1.4cm]90:$T_{A1}$}] {};
	\node (building_rect) at (1.25,-1.5) [airzonestyle,label={[label distance=-1cm]90:Zone 2},label={[label distance =-1.4cm]90:$T_{A2}$}] {};
	\node (hvac_rect) at (0,0.75) [hvacstyle] {HVAC};
	
	% Arrows.
	\draw[flow, HeatFlow] (-3.5, 1) -- (-2.75, 0) node[midway,below left,flowlabel] {Radiation};
	\draw[flow, HeatFlow] (-4.5, -1.5) -- (-2.75, -1.5) node[midway,below,flowlabel] {Convection} node[midway,above,flowlabel] {Ambient};
	\draw[flow, HeatFlow] (-1.8, -2.4) -- ++(0, 0.8) node[align=center,midway,right,flowlabel,xshift=-0.05cm, font=\tiny\itshape] {Heat\\Generation};
	\draw[flow, HeatFlow] (1.8, -2.4) -- ++(0, 0.8) node[align=center,midway,right,flowlabel,xshift=-0.05cm] {};
	\draw[flow, HeatFlow, <->] (-0.5, -2.8) -- ++(1, 0) node[align=center,above,right,flowlabel,yshift=0.05cm, font=\tiny\itshape] {Heat Transfer \\ Between Zones};
	\draw[flow, HeatFlow, <->] (-0.8, -1.6) -- ++(0.6, 0) node[align=center,above,right,flowlabel,yshift=0.05cm] {};
	\draw[flow, HeatFlow, <->] (0.8, -1.6) -- ++(-0.6, 0) node[align=center,above,right,flowlabel,yshift=0.05cm] {};
	\draw[flow, CoolFlow] (-0.75, -0.8) node[right,flowlabel,yshift=0.5cm] {} -- ++(0, 1.2);
	\draw[flow, CoolFlow] (0.75, -0.8) node[left,flowlabel,yshift=0.5cm] {} -- ++(0, 1.2);
	\draw[flow, OtherFlow, dashed] (-0.5, 2) node[above,flowlabel] {Power} -- ++(0, -1);
	\draw[flow, HeatFlow] (hvac_rect.east) -- ++(1, 0) -- ++(0, 1) node[align=center,midway,right,flowlabel] {Rejection \\ to Ambient};
	
	% Labels.
	\node at (0,0) [above, flowlabel] {Cooling};
	\node at (1.6,-0.5) [below, flowlabel] {Air};
	\node at (1.3,-0.5) [above, flowlabel] {Mass};
	\node at (-3.5,-1.2) [above, flowlabel] {$T_a$};
	\node at (-2,-0.5) [above, flowlabel] {$T_{C1}$};
	\node at (2,-0.5) [above, flowlabel] {$T_{C2}$};
	\node at (-0.4,-0.55) [above, flowlabel] {$\dot Q_{C1}$};
	\node at (0.4,-0.55) [above, flowlabel] {$\dot Q_{C2}$};
	\end{tikzpicture}}
	\begin{align*}
		C_i\dfrac{dT_{i}}{dt} &= -H_{i}(T_i-T_a)-\sum_{j \neq i}\beta_{ij}(T_i-T_j)-\dot Q_{ci}(v_i)+\dot Q_{oi}
		\end{align*} 
		Nonlinearity between the valve ($v_i$) to the cooling duty 
		($\dot Q_{ci}$). $x \in \{ T_{A1}, T_{C1}, T_{A2}, T_{C2} \}$, $u \in \{v_1, v_2\}$, and $p$ is the ambient temperature $T_a$ (disturbance).	
\end{frame}

\begin{frame}{Nonlinearity in the plant-model}
	\begin{figure}[!h]
		\centering
		\includegraphics[page = 5,width=0.6\textwidth]{plotshvacnn.pdf}
	\end{figure}
	\vspace{-0.1in}
	\begin{itemize}
		\item Test the performance of linear MPC in regions of high plant-model mismatch.
	\end{itemize}
\end{frame}
	
\begin{frame}{Linear MPC vs NN-based MPC}
	\begin{figure}[!h]
		\centering
		\includegraphics[page = 2, width=0.48\textwidth]{plotshvacnn.pdf} \hfill
		\includegraphics[page = 3, width=0.48\textwidth]{plotshvacnn.pdf}
	\end{figure}
	\begin{itemize}
		\item NN trained on noisy open-loop identification data (5000 samples).
		\item The NN model provides superior closed-loop performance than the linear model.
		\item \textbf{Future work --} \\
		i) Develop identification frameworks to build
		hybrid grey-box + black-box models. \\
		ii) Use neural networks to model unmeasured disturbance dynamics (e.g., heat disturbances in HVAC applications).
	\end{itemize}
\end{frame}

\begin{frame}{References}
\bibliographystyle{abbrvnat}
\bibliography{articles,proceedings,books,unpub}
\end{frame}

\end{document}