\documentclass[xcolor=dvipsnames, 8pt]{beamer} %
%\setbeamertemplate{navigation symbols}{}

\usetheme{SantaBarbara}

\definecolor{black}{HTML}{0A0A0A}
\definecolor{green}{HTML}{0420eb} 
\definecolor{darkgreen}{HTML}{008000}
\definecolor{gold}{HTML}{FFD000}
\setbeamercolor{normal text}{fg=black,bg=white}
\setbeamercolor{alerted text}{fg=red}
\setbeamercolor{example text}{fg=black}
\setbeamercolor{palette primary}{fg=black, bg=gray!20}
\setbeamercolor{palette secondary}{fg=black, bg=gray!20}
\setbeamercolor{palette tertiary}{fg=white, bg=green!80}
\setbeamercolor{block title}{fg=black,bg=gold!40}
\setbeamercolor{frametitle}{fg=white, bg=green!80}
\setbeamercolor{title}{fg=white, bg=green!80}

\newcommand{\svec}{\operatorname{svec}}
\usepackage[version=4]{mhchem}

\usepackage[utf8]{inputenc}

\usepackage{tikz, tikzsettings}
\usepackage{verbatim}
\usepackage{amssymb}
\usepackage{amsmath}
%\usepackage{amsfonts}
\usepackage{algorithmic}
\graphicspath{{./}{./figures/}{./figures/presentation/}}
%\usepackage[version=4]{mhchem}
\usepackage{subcaption}
\usepackage[authoryear,round]{natbib}
%\usepackage{fancyvrb}
\usepackage{color}
\usepackage{colortbl}
\usepackage{xcolor}
\usepackage{physics}
\usepackage{pgfplots}
\usepackage{ragged2e}
%\pgfplotsset{compat=newest} 
%\pgfplotsset{plot coordinates/math parser=false}
%\usepackage{environ}
%\usetikzlibrary{decorations.markings}
%\usetikzlibrary{decorations.pathreplacing}
\usetikzlibrary{shapes,calc,spy, calc, backgrounds,arrows, fit, decorations.pathmorphing, decorations.pathreplacing, matrix}
%\usepackage[absolute,overlay]{textpos}
\usepackage{caption}
\usepackage{graphicx}
%\usepackage[controls=true,poster=first]{animate}


\AtBeginSection[]
{\frame<beamer>{\frametitle{Outline}   
	\tableofcontents[currentsection, currentsection]}
	\addtocounter{framenumber}{-1}}
	%
%{
%	\frame<beamer>{ 
%		\frametitle{Outline}   
%		\tableofcontents[currentsection,currentsubsection]}
		
%}

\makeatother
\setbeamertemplate{footline}
{
	\leavevmode%
	\hbox{%
		\begin{beamercolorbox}[wd=.3\paperwidth,ht=2.25ex,dp=1ex,center]{author in head/foot}%
			\usebeamerfont{author in head/foot}\insertshortauthor
		\end{beamercolorbox}%
		\begin{beamercolorbox}[wd=.6\paperwidth,ht=2.25ex,dp=1ex,center]{title in head/foot}%
			\usebeamerfont{title in head/foot}\insertshorttitle
		\end{beamercolorbox}%
		\begin{beamercolorbox}[wd=.1\paperwidth,ht=2.25ex,dp=1ex,center]{date in head/foot}%
			\insertframenumber{} / \inserttotalframenumber\hspace*{1ex}
	\end{beamercolorbox}}%
	\vskip0pt%
}


\title[Nonlinear system identification with neural networks for 
process control]{Nonlinear system identification with neural networks for 
process control}
\date{September 15, 2020}
\author[Kumar and Rawlings]{\large Pratyush Kumar and James B. Rawlings}
\institute[UCSB]{
	\begin{minipage}{4in}
		\vspace{-10pt}
		\centering
		\raisebox{-0.1\height}{\includegraphics[width=0.25\textwidth]{UCSB_seal}}
		%\hspace*{.2in}
		%\raisebox{-0.5\height}{\includegraphics[width=0.25\textwidth]{jci_logo}}
	\end{minipage}
	\vspace{10pt}
	\newline
	{\large Department of Chemical Engineering}
	\vspace{10pt}
	\newline
	{\large TWCCC Meeting (Virtual)}}
	%\vspace{10pt}
	%\newline
	%{\large Austin, TX}}

\begin{document}

\frame{\titlepage}

% MPC intro frame
\begin{frame}{System identification for model predictive control (MPC)}

MPC requires a dynamic plant model ($x^+ = f(x, u)$) for use in online optimization.  

	\begin{block}{Standard industrial practice}
		\begin{itemize}
			\item Estimate a linear model from data and use integrating disturbance models to maintain zero offset 
			in controlled measurements.
			\item For highly nonlinear plants, the closed-loop controller performance can deteriorate due to significant model mismatch. 
		\end{itemize}
	\end{block}
	
	\begin{block}{Neural networks and system identification\footnote[frame]{\cite{hussain:1999, draeger:engell:ranke:1995, prasad:bequette:2003, eaton:rawlings:ungar:1994, bhat:minderman:mcavoy:wang:1990}}}
		\begin{itemize}
			\item \textbf{Black-box identification:} Estimate the entire predictive model ($x^+ =  f(x, u)$) from actuators to
			sensors using the data collected from open-loop identification experiments.
			\item \textbf{Hybrid model identification:} Model the dynamics from actuators to sensors using first-principle knowledge, and use
			neural networks to estimate the mismatch or any unmeasured disturbance dynamics.  
		\end{itemize}
	\end{block}
\end{frame}

\begin{frame}{Nonlinear HVAC example}
	\begin{align*}
	\dot{x} = Ax + Bg(u) + B_pp
	\end{align*}
	Nonlinearity between the valve (actuator) to the cooling duty. $x \in \{ T_{A1}, T_{C1}, T_{A2}, T_{C2} \}$, $u \in \{v_1, v_2\}$,
	and $p$ is the ambient temperature $T_a$ (disturbance).
	\pause
	\begin{figure}[!h]
		\centering
		\includegraphics[page = 5,width=0.6\textwidth]{plotshvacnn.pdf}
	\end{figure}
	\alert{Objective: Test the performance of linear MPC in regions of high plant-model mismatch.}
	\end{frame}
	
\begin{frame}{Linear MPC vs NN based MPC}
	\begin{figure}[!h]
		\centering
		\includegraphics[page = 2, width=0.48\textwidth]{plotshvacnn.pdf} \hfill
		\includegraphics[page = 3, width=0.48\textwidth]{plotshvacnn.pdf}
	\end{figure}
	
	\begin{itemize}
		\item NN trained on noisy open-loop system identification data (5000 samples).
		\item Change in ambient temperature at 10 hours, and setpoint changes in air-zone 
		temperatures at 50 and 100 hours. 
		\item The NN model provides superior closed-loop performance than the linear model.
	\end{itemize}
\end{frame}

	\begin{frame}{References}
\bibliographystyle{abbrvnat}
\bibliography{articles,proceedings,books,unpub}
\end{frame}

\end{document}