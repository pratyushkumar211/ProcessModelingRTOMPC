\documentclass[xcolor=dvipsnames, 8pt]{beamer} %
%\setbeamertemplate{navigation symbols}{}

\usetheme{SantaBarbara}

\definecolor{black}{HTML}{0A0A0A}
\definecolor{green}{HTML}{0420eb} 
\definecolor{darkgreen}{HTML}{008000}
\definecolor{gold}{HTML}{FFD000}
\setbeamercolor{normal text}{fg=black,bg=white}
\setbeamercolor{alerted text}{fg=red}
\setbeamercolor{example text}{fg=black}
\setbeamercolor{palette primary}{fg=black, bg=gray!20}
\setbeamercolor{palette secondary}{fg=black, bg=gray!20}
\setbeamercolor{palette tertiary}{fg=white, bg=green!80}
\setbeamercolor{block title}{fg=black,bg=gold!40}
\setbeamercolor{frametitle}{fg=white, bg=green!80}
\setbeamercolor{title}{fg=white, bg=green!80}

\newcommand{\svec}{\operatorname{svec}}
\usepackage[version=4]{mhchem}

\usepackage[utf8]{inputenc}

\usepackage{tikz, tikzsettings}
\usepackage{verbatim}
\usepackage{amssymb}
\usepackage{amsmath}
%\usepackage{amsfonts}
\usepackage{algorithmic}
\graphicspath{{./}{./figures/}{./figures/presentation/}}
%\usepackage[version=4]{mhchem}
\usepackage{subcaption}
\usepackage[authoryear,round]{natbib}
%\usepackage{fancyvrb}
\usepackage{color}
\usepackage{colortbl}
\usepackage{xcolor}
\usepackage{physics}
\usepackage{pgfplots}
\usepackage{ragged2e}
%\pgfplotsset{compat=newest} 
%\pgfplotsset{plot coordinates/math parser=false}
%\usepackage{environ}
%\usetikzlibrary{decorations.markings}
%\usetikzlibrary{decorations.pathreplacing}
\usetikzlibrary{shapes,calc,spy, calc, backgrounds,arrows, fit, decorations.pathmorphing, decorations.pathreplacing, matrix}
%\usepackage[absolute,overlay]{textpos}
\usepackage{caption}
\usepackage{graphicx}
%\usepackage[controls=true,poster=first]{animate}


\AtBeginSection[]
{\frame<beamer>{\frametitle{Outline}   
	\tableofcontents[currentsection, currentsection]}
	\addtocounter{framenumber}{-1}}
	%
%{
%	\frame<beamer>{ 
%		\frametitle{Outline}   
%		\tableofcontents[currentsection,currentsubsection]}
		
%}

\makeatother
\setbeamertemplate{footline}
{
	\leavevmode%
	\hbox{%
		\begin{beamercolorbox}[wd=.3\paperwidth,ht=2.25ex,dp=1ex,center]{author in head/foot}%
			\usebeamerfont{author in head/foot}\insertshortauthor
		\end{beamercolorbox}%
		\begin{beamercolorbox}[wd=.6\paperwidth,ht=2.25ex,dp=1ex,center]{title in head/foot}%
			\usebeamerfont{title in head/foot}\insertshorttitle
		\end{beamercolorbox}%
		\begin{beamercolorbox}[wd=.1\paperwidth,ht=2.25ex,dp=1ex,center]{date in head/foot}%
			\insertframenumber{} / \inserttotalframenumber\hspace*{1ex}
	\end{beamercolorbox}}%
	\vskip0pt%
}


\title[Data-based completion of initial grey-box models for process control]{Data-based completion of initial grey-box models for process control}
\date{September 15, 2020}
\author[Kumar and Rawlings]{\large Pratyush Kumar and James B. Rawlings}
\institute[UCSB]{
	\begin{minipage}{4in}
		\vspace{-10pt}
		\centering
		\raisebox{-0.1\height}{\includegraphics[width=0.25\textwidth]{UCSB_seal}}
		%\hspace*{.2in}
		%\raisebox{-0.5\height}{\includegraphics[width=0.25\textwidth]{jci_logo}}
	\end{minipage}
	\vspace{10pt}
	\newline
	{\large Department of Chemical Engineering}
	\vspace{10pt}
	\newline
	{\large TWCCC Meeting (Virtual)}}
	%\vspace{10pt}
	%\newline
	%{\large Austin, TX}}

\begin{document}

\frame{\titlepage}

% MPC intro frame
\begin{frame}{Grey-box models and undermodeling}
\begin{block}{Process models}
\begin{itemize}
	\item Most first principle based grey-box modeling of chemical processes
	are incomplete. For e.g, due to unknown reactions, incorrect 
	quasi steady state assumption, etc.
	\item Routine operational data can be used in conjuction 
	with first principle knowledge to improve the prediction 
	quality of the initial grey-box models.
\end{itemize}
\end{block}

\begin{block}{Motivating example}
	\begin{itemize}
		\item Consider the reaction system
		\begin{center}
			\ce{A ->[k1] B},\hspace{0.2in} \ce{B ->[k2] C}, \hspace{0.2in} \ce{2B ->[k3] D}
		\end{center}
		\item A is the primary reactant, B is an intermediate, 
		C and D are products.
		\item Depending on the ratios of rate constants and measured 
		concentrations, the reaction model may not be 
		completely characterized with only grey-box models.
	\end{itemize}
\end{block}
\end{frame}

\begin{frame}{Completing grey-box models from operational data}
	\begin{block}{Augment grey-box models with 
				  black-box parameterized models}
		\begin{align*}
			\begin{bmatrix}
			  x_G \\
			  x_B
			\end{bmatrix}^+ = \begin{bmatrix}
			  f_G(x_G, \theta_G) + g_B(x_G, x_B, \theta_B)\\
			  f_B(x_G, x_B, \theta_B) \\
			\end{bmatrix}, \quad
			y = h_G(x_G) + h_B(x_G, x_B)
		  \end{align*}
		  \begin{itemize}
			\item $f_G(\cdot)$ and $h_G(\cdot)$ is the initial grey-box model. 
			$g_B(\cdot)$, $f_B(\cdot)$, and $h_B(\cdot)$ is the parameterized black-box model (for e.g., a neural network).
			\item $x_G$ and $x_B$ are the grey-box and black-box model states
			respectively.
		\end{itemize}
	\end{block}
	\begin{block}{System identification framework}
		\begin{itemize}
			\item Minimize the following 
			multi-step-ahead prediction-error to compute the 
			parameters $\theta_B$
			\begin{align*}
		\min_{x_G(0), \ x_B(0), \ \theta_B} \sum_{k=0}^{N_s-1} (\hat{y}(k) - y(k))^2
			\end{align*}
			\item $x_G(0)$, $x_B(0)$ are the initial states at the 
			start of the forecasting horizon and $\hat{y}$ is the model forecast.
		\end{itemize}
	\end{block}
\end{frame}
	
\begin{frame}{Reaction system example}
	\begin{itemize}
		\item Ethene reacts with chlorine to form Dichloroethane and Dichlorobutane.
	\end{itemize}
	\begin{align*}
		\ce{3C2H4 + 2Cl2 -> C2H4Cl2 + C4H8Cl2}
	\end{align*} 
	\begin{itemize}
		\item The reaction mechanism is -
	\end{itemize}
	\small{\begin{align*}
		\ce{Cl2 + M &<=>[k1][k_{-1}] 2Cl* + M}\\
		\ce{Cl* + C2H4 &->[k2] C2H4Cl*}\\
		\ce{C2H4Cl* + Cl2 &->[k3] C2H4Cl2 + Cl*}\\
		\ce{C2H4Cl* + C2H4Cl* &->[k4] C4H8Cl2}\\
		\ce{Cl* +  C2H4Cl* &->[k5] C2H4Cl2}
	\end{align*}}
\normalsize{\begin{itemize}
	\item The rate constants are such that a quasi steady state assumption (QSSA) on the species $\ce{Cl*}$ and $\ce{C2H4Cl*}$ is incorrect, and neural networks are used to improve the predictions of a  
	grey-box model based on the QSSA assumption.
\end{itemize}}
\end{frame}

\begin{frame}{Data requirements and model validation}
	Show two plots. \\
	On the left, show a performance metric
	that shows the prediction of the overall model improves with data
	relative to the grey-box model prediction. \\
	On the right, show the model predictions vs actual states. \\
	At the bottom, add two bullets discussing future work. \\
\end{frame}

%\begin{frame}[allowframebreaks]{References}
%\bibliographystyle{abbrvnat}
%\bibliography{articles,proceedings,books,unpub}
%\end{frame}

\end{document}