\documentclass[10pt]{article}
\usepackage[utf8]{inputenc}
\usepackage{amsmath}
\usepackage{authblk}
\usepackage{amssymb}
%\usepackage{lucidbry}
\usepackage{graphicx}
\usepackage{latexsym}
\usepackage{subcaption}
\usepackage{physics}
\usepackage[dvipsnames]{xcolor}
\usepackage{tikz,tikzsettings}
\usepackage{verbatim}
\usetikzlibrary{arrows,shapes}
\usepackage{amssymb}
\usepackage{amsmath}
\usepackage{verbatim}
\usepackage{float}
\usepackage{units}
\usepackage[round]{natbib}
\usepackage[english]{babel}
\bibliographystyle{plainnat}
%\setcitestyle{numeric,open={(},close={)}} \usepackage{color}
%\usepackage{colortbl}
\usetikzlibrary{decorations.markings}
\usetikzlibrary{decorations.pathreplacing}
\usetikzlibrary{shapes,calc,backgrounds,arrows,fit,decorations.pathmorphing,matrix}
\usepackage[margin=0.79in]{geometry}

\graphicspath{{./}{./figures/}{./figures/paper/}}

\newcommand{\Is}{\bm{\mathrm{I}}} % Resources.
\newcommand{\Js}{\bm{\mathrm{J}}} % Units.
\newcommand{\Ks}{\bm{\mathrm{K}}} % Resources.
\newcommand{\Ms}{\bm{\mathrm{M}}} % Transition catalog.
\newcommand{\Ns}{\bm{\mathrm{N}}} % Transition counter.

\newcommand{\useq}{\mathbf{u}} \newcommand{\xseq}{\mathbf{x}}
\newcommand{\bbR}{\mathbb{R}} \newcommand{\bbW}{\mathbb{W}}
\newcommand{\bbU}{\mathbb{U}} \newcommand{\bbI}{\mathbb{I}}
\newcommand{\bbX}{\mathbb{X}} \newcommand{\bbP}{\mathbb{P}}

\title{Hybrid process modeling strategies for real time optimization and control}
\author{Pratyush Kumar}
\date{\today}

\begin{document}

\maketitle

\section{Process modeling}
We consider the problem of system identification of the following nonlinear
plant:
\begin{align*}
  x_p^+ &= f_p(x_p, u, w) \\
  y &= h_p(x_p) + v
\end{align*}
in which $x_p \in \bbR^{n_p}$ is the plant state, $u \in \bbU \subset \bbR^m$ is
the manipulated control input, $y \in \bbR^p$ is the measurement, $w$ is the
process noise, and $v$ is the measurement noise. The objective of developing the
model is to use it subsequently in model predictive control (MPC). Therefore, we
consider several types of grey-box, hybrid, and black-box models and analyze
their predictive capabilities as well as their usefulness for online
optimization in MPC. 

\section{Model architectures for system identification}
We describe several process modeling frameworks which determine parameters in
their proposed model architecture by minimizing the multi-step ahead prediction
error of the model on training data. A nonlinear optimization problem is solved
for this identification step. The estimated parameters define the model, which
is then used subsequently to perform model validation and for use in MPC.

\subsection{Grey-box model}
We first consider the grey-box modeling strategy, which is the current
industrial state-of-the-art system identification technique. A parameterized
model is defined based on first-principle knowledge about the plant, and
parameters in the model are estimated from training data by solving the
prediction-error minimization problem. The grey-box model is typically defined
in continuous time and is of the following form:

\begin{align*}
  \dot{x}_g &= f_g(x_g, u, \theta_g) \\
  y &= h_g(x_g, \theta_g)
\end{align*}

The following nonlinear program (NLP) is solved to identify the parameters
$\theta_g$:

\begin{align} \label{eq:gb_training}
  \underset{x_g^i(0), \ \theta_g}{\textnormal{min}} \sum_{i=1}^{N_{tr}} \sum_{k=1}^{N_t} 
  \dfrac{1}{N_tN_{tr}} &\norm{\tilde{y}^i(k) - y^i(k)}^2_{R_v^{-1}}  \\
  \textnormal{subject to} \quad \quad x_g^{i+} &= f_{gd}(x_g^i, u^i, \theta_g) \nonumber \\
   \tilde{y}^i &= h_g(x_g^i, \theta_g) \nonumber
\end{align}

in which $\theta_g$ and $x_g^i(0)$ are the decision variables in the
optimization problem, which are parameters in the grey-box model and initial
state used for forecasting the predicted measurements respectively. We assume
that multiple trajectories can be used in the training data, and the superscript
$i$ is used to denote the trajectory number. We use $N_{tr}$ to denote the total
number of trajectories in training data and $N_t$ to denote the number of
samples in each trajectory. The function $f_{g}(\cdot)$ is discretized at the
measurement sample time to obtain the function $f_{gd}(\cdot)$ that is used to
make forecasts in the above optimization.

\subsection{Hybrid model}

Often in industrial applicationns, first-principle knowledge about the plant is
incomplete. Hence, grey-box system identification and subsequent use of the
grey-box model in real-time optimization can lead to suboptimal economic
performance. We propose to augment the grey-box model using a neural network
(NN) such that the overall hybrid model can capture any unknown process dynamics
in the plant, therefore leading to improved economic performance when used
subsequently in real-time optimization. We define this hybrid model as follows:

\begin{align*}
  x_g^+ &= f_{gd}(x_g, u, \theta_g) + f_N(z, u, \theta_N) \\
  z^+ &= f_z(x_g, z, u) \\
  y &= h_g(x_g, \theta_g)
\end{align*}
in which $f_N(\cdot)$ is a NN and $\theta_N$ denotes the parameters in the
network. We use $z$ to denote the vector of past measurements and control
inputs. This vector is considered as an additional state in the overall dynamic
model and is defined as follows:
\begin{align} \label{eq:pastyu}
  \mathbf{y}_{k-N_p:k-1} &= [y(k-N_p)', ... \ , y(k-1)']' \nonumber \\
  \mathbf{u}_{k-N_p:k-1} &= [u(k-N_p)', ... \ , u(k-1)']' \nonumber \\
  z(k) &= [\mathbf{y}_{k-N_p:k-1}', \mathbf{u}_{k-N_p:k-1}']'
\end{align}
in which $N_p$ is the number of past measurements and control inputs used. The
function $f_z(\cdot)$ describes the dynamics of the state $z$ and its structure
is defined as follows: 
\begin{align*}
  f_z(x_g, z, u) = \begin{bmatrix}
    y(k-N_p+1) \\
    y(k-N_p+2) \\
    \vdots \\
    h_g(x_g, \theta_g) \\
    u(k-N_p+1) \\ 
    u(k-N_p+2) \\
    \vdots \\
    u(k)
  \end{bmatrix}
\end{align*}
We assume that the parameters in the grey-box model are first estimated by
solving the NLP \eqref{eq:gb_training}, and the following similar NLP is solved
to determine parameters ($\theta_N$) in the NN.

\begin{align} \label{eq:hybrid_training}
  \underset{\theta_N}{\textnormal{min}} \sum_{i=1}^{N_{tr}} \sum_{k=1}^{N_t} 
  \dfrac{1}{N_tN_{tr}} &\norm{\tilde{y}^i(k) - y^i(k)}^2  \\
  \textnormal{subject to} \quad \quad x_g^{i+} &= f_{gd}(x_g^i, u^i, \theta_g) + f_z(z^i, u^i, \theta_N) \nonumber \\
  z^{i+} &= f_z(x_g^i, z^i, u^i) \nonumber \\
   \tilde{y}^i &= h_g(x_g^i, \theta_g) \nonumber
\end{align}
The stochastic gradient descent algorithm Adam is used to solve the above
optimization using the software tensorflow. Note that we do not estimate the
initial grey-box state for each trajectory in the optimization problem, and fix
those states heuristically to make forecasts during the optimization process.
The control inputs and measurements are scaled as $u := (u -
u_{\textnormal{MEAN}})/u_{\textnormal{STD}}$ and $y := (y -
y_{\textnormal{MEAN}})/y_{\textnormal{STD}}$, in which $u_{\textnormal{MEAN}}$
and $y_{\textnormal{MEAN}}$ are the respective means computed on training data.
Similarly, $u_{\textnormal{STD}}$ and $y_{\textnormal{STD}}$ are the respective
standard deviations. Due to this scaling, we do not consider the penalty
$R_v^{-1}$ in the training objective.

\subsection{Black-box models}
Next, we consider black-box system identification strategies which do not impose
any a-prior knowledge about the plant and develop the dynamic model purely from
data. We consider two types of black-box models in this section: NNs and dynamic
models based on the Koopman operator theory. The latter type of model is
considered based on the motivation that it imposes a linear structure on the
dynamics in a high-dimensional space, therefore leading to a linear or quadratic
program to be solved online in MPC.

\subsubsection{Neural network}
The black-box NN uses the vector of past measurements and control inputs $z$
\eqref{eq:pastyu} as the state in the dynamic model and is structured as follows
\begin{align} \label{eq:bbnn_model}
  z^+ &= f_z(z, u) \\ 
  y &= h_N(z, \theta_N) \nonumber
\end{align}
in which $h_N(\cdot)$ is a NN and $\theta_N$ are the parameters in the network.
The function $f_z(\cdot)$ denotes the dynamics of the state $z$ and is defined
as 

\begin{align*}
  f_z(z, u) = \begin{bmatrix}
    y(k-N_p+1) \\
    y(k-N_p+2) \\
    \vdots \\
    h_N(z, \theta_N) \\
    u(k-N_p+1) \\ 
    u(k-N_p+2) \\
    \vdots \\
    u(k)
  \end{bmatrix}
\end{align*}

\subsubsection{Koopman operator model}

The Koopman model represents the dynamics as a linear system in a
high-dimensional state-space that is a nonlinear transformation of the original
state-space. This transformation is chosen such that the input-output behavior
is linear. The overall dynamic model is represented as:

\begin{align} \label{eq:koop_model}
  x_{kp} &= [y', z', f_N(y, z, \theta_N)']' \\
  x_{kp}^+ &= Ax_{kp} + Bu \nonumber \\
  y &= [I, \ 0] x_{kp} \nonumber
\end{align}
in which $x_{kp} \in \bbR^N$ is the state in a high-dimensional state-space,
$f_N(\cdot)$ is a NN used for the nonlinear transformation, and $\theta_N$
denotes the parameters in the network.

\subsubsection{Training}
The training optimization problem for both the black-box NN and the Koopman
operator model is to minimize the mean-squared-error (MSE) similar to the
nonlinear program \eqref{eq:hybrid_training}. The model equality constraints are
replaced with the equations \eqref{eq:bbnn_model} and \eqref{eq:koop_model} for
the black-box and Koopman operator model respectively. The measurements and
control inputs are scaled using the mean and standard deviation computed from
the training data. The initial state to make model predictions in the
optimization process is set based on a window of past measurements and control
inputs in the training data.

\end{document}
