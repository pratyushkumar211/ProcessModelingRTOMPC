\documentclass[xcolor=dvipsnames, 8pt]{beamer} %
%\setbeamertemplate{navigation symbols}{}

\usetheme{SantaBarbara}

\definecolor{black}{HTML}{0A0A0A}
\definecolor{red}{HTML}{e00404} 
%\definecolor{violet}{HTML}{231A97}

%\definecolor{darkgreen}{HTML}{008000}
\definecolor{gold}{HTML}{FFD000}
\setbeamercolor{normal text}{fg=black,bg=white}
\setbeamercolor{alerted text}{fg=red}
\setbeamercolor{example text}{fg=black}
\setbeamercolor{palette primary}{fg=black, bg=gray!20}
\setbeamercolor{palette secondary}{fg=black, bg=gray!20}

\setbeamercolor{palette tertiary}{fg=white, bg=red!80}
\setbeamercolor{block title}{fg=black,bg=gold!40}
\setbeamercolor{frametitle}{fg=white, bg=red!80}
\setbeamercolor{title}{fg=white, bg=red!80}


\usepackage[utf8]{inputenc}

\usepackage{tightlist}
\usepackage{tikz, tikzsettings}
\usepackage{verbatim}
\usepackage{amssymb}
\usepackage{amsmath}
\usepackage{amsfonts}
\pdfmapfile{+sansmathaccent.map} % Fix done for making the talk in ubuntu.
\usepackage{algorithmic}
\graphicspath{{./}{./figures/}{./figures/presentation/}}

%\usepackage[version=4]{mhchem}
\usepackage{subcaption}
\usepackage[authoryear,round]{natbib}
%\usepackage{fancyvrb}
\usepackage{color}
\usepackage{colortbl}
\usepackage{xcolor}
%\usepackage{physics}
\usepackage{pgfplots}
\usepackage{ragged2e}
%\pgfplotsset{compat=newest} \pgfplotsset{plot coordinates/math parser=false}
%\usepackage{environ} \usetikzlibrary{decorations.markings}
%\usetikzlibrary{decorations.pathreplacing}
\usetikzlibrary{shapes,calc,spy, calc, backgrounds,arrows, fit, decorations.pathmorphing, decorations.pathreplacing, matrix}
%\usepackage[absolute,overlay]{textpos}
\usepackage{caption}
\usepackage{mpcsymbols}
\usepackage{graphicx}
%\usepackage[controls=true,poster=first]{animate}


\AtBeginSection[] {\frame<beamer>{\frametitle{Outline}   
	\tableofcontents[currentsection, currentsection]}
	\addtocounter{framenumber}{-1}}
	%
%{
%	\frame<beamer>{\frametitle{Outline}   
%	    \tableofcontents[currentsection,currentsubsection]}
		
%}
\newcommand{\calert}[1]{\textcolor{blue}{#1}}

\makeatother
\setbeamertemplate{footline}
{\leavevmode%
	\hbox{%
		\begin{beamercolorbox}[wd=.3\paperwidth,ht=2.25ex,dp=1ex,center]{author
		in head/foot}%
			\usebeamerfont{author in head/foot}\insertshortauthor
		\end{beamercolorbox}%
		\begin{beamercolorbox}[wd=.6\paperwidth,ht=2.25ex,dp=1ex,center]{title
		in head/foot}%
			\usebeamerfont{title in head/foot}\insertshorttitle
		\end{beamercolorbox}%
		\begin{beamercolorbox}[wd=.1\paperwidth,ht=2.25ex,dp=1ex,center]{date in
		head/foot}%
			\insertframenumber{} / \inserttotalframenumber\hspace*{1ex}
	\end{beamercolorbox}}%
	\vskip0pt%
}


\title{Case studies on hybrid modeling and application to steady-state process optimization}
\date{October 26, 2021}
\author[Pratyush Kumar]{\large Pratyush Kumar}
\institute[UCSB]{
	\begin{minipage}{4in}
		\vspace{-10pt}
		\centering
		\raisebox{-0.1\height}{\includegraphics[width=0.25\textwidth]{UCSB_seal}}
		%\hspace*{.2in}
		%\raisebox{-0.5\height}{\includegraphics[width=0.25\textwidth]{jci_logo}}
	\end{minipage}
	\vspace{10pt}
	\newline
	{\large Department of chemical engineering}
	\vspace{10pt}
	\newline
	{\large Group Meeting Presentation}}

\begin{document}

\frame{\titlepage}

%\section{Background}


\begin{frame}{Hybrid process modeling}

	\begin{columns}
	\column{\textwidth}

	\begin{block}{Motivation}
		\begin{itemize}
		\item A dynamic model of the process is required in feedback
		control applications to achieve operational objectives. \pause
		\item Usual first principles based \textcolor{blue}{grey-box models are
		incomplete}, and \textcolor{blue}{machine learning methods can provide a
		flexible approach to improve the performance of the grey-box
		models} for use in real time optimization (RTO).
		\end{itemize}
	  \end{block}
	  \pause
	  \bigskip
	\begin{block}{Literature}
	  \begin{itemize}
	  \item Use neural networks to approximate some state dependent process
	  parameters \footnote[frame]{\cite{psichogios:ungar:1992}} or to
	  approximate specific functions, e.g, unknown reaction rate laws, vapor
	  liquid equilibrium relationships,
	  \footnote[frame]{\cite{lovelett:avalos:kevrekidis:2019,
	  chen:ierapetritou:2020, bangi:kwon:2020}} in the grey-box models.	
	  \end{itemize}
	\end{block}
  
\end{columns}
\end{frame}

\begin{frame}{Process model types}

	Hi.

\end{frame}

\begin{frame}{Black-Box}

	Hi.

\end{frame}

\begin{frame}{Hybrid model with all grey-box states}

	Hi.

\end{frame}

\begin{frame}{Hybrid model with only measured grey-box states}

	Hi.

\end{frame}

\begin{frame}{Sample measurement data}

	Hi.

\end{frame}

\begin{frame}{Model validation}

	Hi.

\end{frame}

\begin{frame}{Analysis of reaction rates}

	\begin{figure}
		\centering
		\includegraphics[page=10, height=0.7\textheight, 
		width=0.48\textwidth]{reac_plots.pdf}
		\includegraphics[page=11, height=0.7\textheight, 
		width=0.48\textwidth]{reac_plots.pdf}
	\end{figure}

\end{frame}

\begin{frame}{Analysis of reaction rates}
	
	\begin{figure}
		\centering
		\includegraphics[page=12, height=0.7\textheight, 
		width=0.48\textwidth]{reac_plots.pdf}
		\includegraphics[page=11, height=0.7\textheight, 
		width=0.48\textwidth]{reac_plots.pdf}
	\end{figure}
	
\end{frame}

\begin{frame}{Steady-state optimization problem}

	Hi.

\end{frame}

\begin{frame}{Steady-state cost curves (Cost Type 1)}

	Hi.

\end{frame}

\begin{frame}{Optimization analysis (Cost Type 1)}

	Hi.

\end{frame}

\begin{frame}{Steady-state cost curves (Cost Type 2)}

	Hi.

\end{frame}

\begin{frame}{Optimization analysis (Cost Type 2)}

	Hi.

\end{frame}

\begin{frame}{Conclusions}

	Hi.

\end{frame}

\begin{frame}{References}
\bibliographystyle{abbrvnat}
\bibliography{articles,proceedings,books,unpub, resgrppub}
\end{frame}

\end{document}