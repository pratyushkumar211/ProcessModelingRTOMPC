\documentclass[xcolor=dvipsnames, 8pt]{beamer} %
%\setbeamertemplate{navigation symbols}{}

\usetheme{SantaBarbara}

\definecolor{black}{HTML}{0A0A0A}
\definecolor{red}{HTML}{e00404} 
%\definecolor{violet}{HTML}{231A97}

\definecolor{blue}{HTML}{0647A8}
\definecolor{darkgreen}{HTML}{008000}

\definecolor{gold}{HTML}{FFD000}
\setbeamercolor{normal text}{fg=black,bg=white}
\setbeamercolor{alerted text}{fg=red}
\setbeamercolor{example text}{fg=black}
\setbeamercolor{palette primary}{fg=black, bg=gray!20}
\setbeamercolor{palette secondary}{fg=black, bg=gray!20}

\setbeamercolor{palette tertiary}{fg=white, bg=red!80}
\setbeamercolor{block title}{fg=black,bg=gold!40}
\setbeamercolor{frametitle}{fg=white, bg=red!80}
\setbeamercolor{title}{fg=white, bg=red!80}


\usepackage[utf8]{inputenc}

\usepackage{tightlist}
\usepackage{tikz, tikzsettings}
\usepackage{verbatim}
\usepackage{amssymb}
\usepackage{amsmath}
\usepackage{amsfonts}
\pdfmapfile{+sansmathaccent.map} % Fix done for making the talk in ubuntu.
\usepackage{algorithmic}
\graphicspath{{./}{./figures/}{./figures/presentation/}}

%\usepackage[version=4]{mhchem}
\usepackage{subcaption}
\usepackage[authoryear,round]{natbib}
%\usepackage{fancyvrb}
\usepackage{color}
\usepackage{colortbl}
\usepackage{xcolor}
%\usepackage{physics}
\usepackage{pgfplots}
\usepackage{ragged2e}
%\pgfplotsset{compat=newest} \pgfplotsset{plot coordinates/math parser=false}
%\usepackage{environ} \usetikzlibrary{decorations.markings}
%\usetikzlibrary{decorations.pathreplacing}
\usetikzlibrary{shapes,calc,spy, calc, backgrounds,arrows, fit, decorations.pathmorphing, decorations.pathreplacing, matrix}
%\usepackage[absolute,overlay]{textpos}
\usepackage{caption}
\usepackage{mpcsymbols}
\usepackage{graphicx}
%\usepackage[controls=true,poster=first]{animate}


\AtBeginSection[] {\frame<beamer>{\frametitle{Outline}   
	\tableofcontents[currentsection, currentsection]}
	\addtocounter{framenumber}{-1}}
	%
%{
%	\frame<beamer>{\frametitle{Outline}   
%	    \tableofcontents[currentsection,currentsubsection]}
		
%}
\newcommand{\calert}[1]{\textcolor{blue}{#1}}

\makeatother
\setbeamertemplate{footline}
{\leavevmode%
	\hbox{%
		\begin{beamercolorbox}[wd=.3\paperwidth,ht=2.25ex,dp=1ex,center]{author
		in head/foot}%
			\usebeamerfont{author in head/foot}\insertshortauthor
		\end{beamercolorbox}%
		\begin{beamercolorbox}[wd=.6\paperwidth,ht=2.25ex,dp=1ex,center]{title
		in head/foot}%
			\usebeamerfont{title in head/foot}\insertshorttitle
		\end{beamercolorbox}%
		\begin{beamercolorbox}[wd=.1\paperwidth,ht=2.25ex,dp=1ex,center]{date in
		head/foot}%
			\insertframenumber{} / \inserttotalframenumber\hspace*{1ex}
	\end{beamercolorbox}}%
	\vskip0pt%
}


\title{A case study on hybrid modeling and application to steady-state 
process optimization}
\date{October 26, 2021}
\author[Pratyush Kumar]{\large Pratyush Kumar}
\institute[UCSB]{
	\begin{minipage}{4in}
		\vspace{-10pt}
		\centering
		\raisebox{-0.1\height}{\includegraphics[width=0.25\textwidth]{UCSB_seal}}
		%\hspace*{.2in}
		%\raisebox{-0.5\height}{\includegraphics[width=0.25\textwidth]{jci_logo}}
	\end{minipage}
	\vspace{10pt}
	\newline
	{\large Department of chemical engineering}
	\vspace{10pt}
	\newline
	{\large Group meeting presentation}}

\begin{document}

\frame{\titlepage}


\begin{frame}{Outline} 
	\tableofcontents 
\end{frame}

\section{Introduction and process model types}
\begin{frame}{System identification}

	\begin{columns}
	\column{\textwidth}

	\begin{block}{Motivation}
		\begin{itemize}
	\item A dynamic model of the process is required in feedback
		control applications to achieve operational objectives. \pause
	\medskip
	\item The model can be used in several places in the process operations 
	hierarchy such as the \textcolor{blue}{real-time optimization layer} 
		\footnote[frame]{\cite{darby:nikolaou:jones:nicholson:2011}} or 
	\textcolor{blue}{model predictive control} 
	\footnote[frame]{\cite{qin:badgwell:2003, 
	lahiri:2017}}.
	\end{itemize}
	\end{block}
	  \pause
	  \bigskip
	\begin{block}{Literature}
	  \begin{itemize}
	  \item System identification is a widely studied problem 
	  \footnote[frame]{\cite{ljung:1999, qin:2006}}, with a variety of methods 
	  such as prediction error minimization and subspace identification.
	  \medskip
	  \item Hybrid process modeling with neural networks is also proposed in 
	 the literature\footnote[frame]{\cite{psichogios:ungar:1992, 
	 lovelett:avalos:kevrekidis:2019, chen:ierapetritou:2020, 
	 bangi:kwon:2020}}, 
	 to approximate some state dependent parameters or 
	 specific functions in grey-box models.	
	  \item Less attention is paid to \textcolor{red}{optimization over the 
	  hybrid process models.} 
	  \end{itemize}
	\end{block}
  
\end{columns}
\end{frame}
	
\begin{frame}{Chemical reactor example}
		
		\centerline{\resizebox{0.5\textwidth}{!}{\input{cstropt}}}
		\pause
		\begin{block}{Plant model}
			\begin{equation*}
				\renewcommand{\arraystretch}{2}
				\begin{bmatrix} 
					\dfrac{dc_A}{dt} \\
					\dfrac{dc_B}{dt} \\
					\dfrac{dc_C}{dt}
				\end{bmatrix} = \begin{bmatrix}
					\begin{array}{ccl}
						\dfrac{Q_f(c_{Af} - c_A)}{V_R} &-& \alert{r_1} 
						\\ 
						-\dfrac{Q_fc_B}{V_R} &+& \alert{r_1} - 3\alert{r_2} \\
						-\dfrac{Q_fc_C}{V_R} &+& \alert{r_2}
					\end{array}
				\end{bmatrix}
			\end{equation*}
	\begin{itemize}
		\item States $x = \begin{bmatrix} c_A, c_B, c_C\end{bmatrix}'$, 
		Measurements $y = \begin{bmatrix} c_A, c_B\end{bmatrix}'$, and control 
		input $u = c_{Af}$. 
		\item Rate laws $r_1 = k_1c_A$ and $r_2 = k_{2f}c^3_B - k_{2b}c_C$.
	\end{itemize}
	\end{block}
			Often the reaction rate laws
${\color{red}r_1}(c_j)$ and ${\color{red} r_2}(c_j)$ are unknown,
so approximate the rate laws with neural networks + data.
\end{frame}

\begin{frame}{Black-Box neural network model}

	\begin{block}{Model}
	
	\begin{equation*}
		z(k) = \begin{bmatrix} y(k-N_p)', ..., y(k-1)',
							   u(k-N_p)', ..., u(k-1)' 
			   \end{bmatrix}'
	\end{equation*}
	\begin{align*}
		z^+ &= f_N(z, u;\theta_N) \\ 
		y &= h_N(z;\theta_N)
	\end{align*}
	\end{block}
	\pause
	\begin{block}{Training optimization problem}
		
	\begin{align*}
		\underset{\theta_N}{\textnormal{min}} \quad \dfrac{1}{N_{tr} 
		N_t} \sum_{i=1}^{N_{tr}}
		\sum_{k=0}^{N_t} & \alert{\norm{y_i(k) - \tilde{y}_i(k)}^2} \\ 
		\textnormal{subject to} \quad z_i^+ & = f_N(z_i, u_i;\theta_N) \\
								  \tilde{y}_i &= h_N(z_i)
	\end{align*}
	This optimization problem is a multi-step ahead prediction error 
	minimization problem.	
	\end{block}

\end{frame}

\begin{frame}{Hybrid model 1 -- With all grey-box states}

\begin{block}{Hybrid-FullGb model}

	\begin{align*}
		\dot{x}_{g} &= f_g(x_g, u;\theta_{r_1}, \theta_{r_2}) \\ 
		y &= h(x_g)
	\end{align*}
	$x_g = \begin{bmatrix} c_A, c_B, c_C\end{bmatrix}$ \\
	$\textcolor{blue}{r_1 = f_{N1}(c_A;\theta_{r_1})}$ and 
	$\textcolor{blue}{r_2 = f_{N2}(c_B, c_C;\theta_{r_2})}$
\end{block}

\pause
\begin{block}{Training optimization problem}
	
	\begin{align*}
		\underset{\theta_{r_1}, 
				  \theta_{r_2}, \theta_{C}}{\textnormal{min}} \quad 
		\dfrac{1}{N_{tr} 
			N_t} \sum_{i=1}^{N_{tr}}
		\sum_{k=0}^{N_t} & \norm{y_i(k) - \tilde{y}_i(k)}^2 \\ 
		\textnormal{subject to} \quad x_{gi}^+ & = f_{gd}(x_{gi}, 
		u;\theta_{r_1}, 
		\theta_{r_2}) \\
		\tilde{y}_i &= h(x_{gi}) \\ 
		\alert{c_{iC}(0)} & \alert{=} \alert{f_{N3}(z_i(0); \theta_C)}\\ 
	\end{align*}
\end{block}

\end{frame}

\begin{frame}{Hybrid model 2 -- With only measured grey-box states}


\begin{block}{Hybrid-PartialGb model}
	
	\begin{align*}
		\dot{x}_{g} &= f_g(x_g, u;\theta_{r_1}, \theta_{r_2}) \\ 
		y &= x_g
	\end{align*}
	$x_g = \begin{bmatrix} c_A, c_B\end{bmatrix}$ \\
	$\textcolor{blue}{r_1 = f_{N1}(c_A;\theta_{r_1})}$ and 
	$\textcolor{blue}{r_2 = f_{N2}(c_B, z;\theta_{r_2})}$
\end{block}

\pause
\begin{block}{Training optimization problem}
	
	\begin{align*}
		\underset{\theta_{r_1}, 
			\theta_{r_2}}{\textnormal{min}} \quad 
		\dfrac{1}{N_{tr} 
			N_t} \sum_{i=1}^{N_{tr}}
		\sum_{k=0}^{N_t} & \norm{y_i(k) - \tilde{y}_i(k)}^2 \\ 
		\textnormal{subject to} \quad x_{gi}^+ & = f_{gd}(x_{gi}, 
		u;\theta_{r_1}, 
		\theta_{r_2}) \\
		\tilde{y}_i &= x_{gi} \\ 
	\end{align*}
\end{block}

\end{frame}

\section{Chemical reactor case study}
\begin{frame}{Sample measurement data}

	\begin{figure}
		\centering
		\includegraphics[page=5, height=0.8\textheight, 
		width=0.6\textwidth]{reac_plots.pdf}
	\end{figure}

\begin{itemize}
\item Use 20 hours of training data (split over 5 trajectories) and 6 hours of 
validation data (1 trajectory). 
\end{itemize}
\end{frame}

\subsection{Model validation}
\begin{frame}{Model validation}

\begin{columns}
	\column{0.6\textwidth}	
	\begin{figure}
		\centering
		\includegraphics[page=6, height=0.8\textheight, 
		width=\textwidth]{reac_plots.pdf}
	\end{figure}
	
	\column{0.4\textwidth}
	
	\begin{block}{Model architectures}
		
		\begin{itemize}
			\item Black-Box-NN: $\begin{bmatrix} 6, 32, 32, 2 \end{bmatrix}$.
			\item Hybrid Full Gb: $f_{N1} = \begin{bmatrix} 1, 8, 1 
			\end{bmatrix}$, $f_{N2} = \begin{bmatrix} 2, 32, 32, 1 
		\end{bmatrix}$.
			\item Hybrid Partial Gb: $f_{N1} = \begin{bmatrix} 1, 8, 1 
				\end{bmatrix}$, $f_{N2} = \begin{bmatrix} 7, 32, 32, 1 
				\end{bmatrix}$.
		\end{itemize}
		
	\end{block}
	
\end{columns}

\end{frame}


\begin{frame}{Steady-state profiles}
	
	\begin{figure}
		\centering
		\includegraphics[page=7, height=0.8\textheight, 
		width=0.6\textwidth]{reac_plots.pdf}
	\end{figure}
	
\end{frame}

\subsection{Reaction rate analysis}
\begin{frame}{Analysis of NN reaction rates -- Over all the generated data}
	
	\centering
	$\%$ Error = $\dfrac{\norm{\textnormal{Rate} - 
	\textnormal{Rate}_{\textnormal{NN}}}}{\norm{\textnormal{Rate}}}$

	\begin{figure}
		\centering
		\includegraphics[page=10, height=0.7\textheight, 
		width=0.48\textwidth]{reac_plots.pdf}
		\includegraphics[page=11, height=0.7\textheight, 
		width=0.48\textwidth]{reac_plots.pdf}
	\end{figure}

\end{frame}

\begin{frame}{Analysis of NN reaction rates -- Over the state-space}
	
	\centering
	Modified error = $\norm{\norm{\textnormal{Rate} - 
						\textnormal{Rate}_{\textnormal{NN}}} - 
						\alpha_1\norm{\textnormal{Rate}} - \alpha_2}, \ 
						\alpha_1 = 10^{-6}, \alpha_2 = 10^{-4}$
	\begin{figure}
		\centering
		\includegraphics[page=12, height=0.7\textheight, 
		width=0.48\textwidth]{reac_plots.pdf}
		\includegraphics[page=13, height=0.7\textheight, 
		width=0.48\textwidth]{reac_plots.pdf}
	\end{figure}
	
\end{frame}

\subsection{Steady-state process optimization}
\begin{frame}{Steady-state optimization problem}


\begin{block}{Optimization problem}
	
\begin{align*}
\underset{u_s}{\textnormal{min}} & \quad \ell(x_s, u_s) \leftarrow 
 \textnormal{Economic cost}\\ 
		x_s &= f(x_s, u_s) \\ 
		\underline{u} \leq &u_s \leq \overline{u}
\end{align*}
Cost types -- \\ 
$\ell(x_s, u_s) = p_A c_{Afs} - p_B c_{Bs}$ (Type 1) \\ 
$\ell(x_s, u_s) = p_A c_{Afs} - p_B c_{Bs} + p_c c_{Cs}$ (Type 2)
\end{block}

$p = \begin{bmatrix} p_A, p_B, p_C \end{bmatrix}$ is vector of cost 
parameters.

\begin{block}{Experiments}
\begin{itemize}
\item Examine steady-state cost curves for a fixed choice of the cost parameter 
($p$).
\item Analyze the optimization solutions (optimal steady-state input and 
economic costs) for a range of parameters $\begin{bmatrix} \underline{p}$, 
$\overline{p} \end{bmatrix}$.
\end{itemize}	
\end{block}

\end{frame}

\begin{frame}{Steady-state cost curves (Cost Type 1)}

\begin{figure}
	\centering
	\includegraphics[page=8, height=0.8\textheight, 
	width=0.6\textwidth]{reac_plots.pdf}
\end{figure}

\end{frame}

\begin{frame}{Optimization analysis (Cost Type 1)}

Randomly generate 500 cost parameters in a range $\begin{bmatrix} 
\underline{p}$, $\overline{p} \end{bmatrix}$ and examine the steady-state 
optimization solutions of the plant and identified model.
\pause
\vspace{-0.05in}
\begin{figure}
	\centering
	\includegraphics[page=14, height=0.7\textheight, 
	width=0.48\textwidth]{reac_plots.pdf}
	\includegraphics[page=15, height=0.7\textheight, 
	width=0.48\textwidth]{reac_plots.pdf}
\end{figure}

\begin{itemize}
	\item $u_s$ = Identified model optimal input and $u^{*}_s$ = plant optimal 
	input.
	\item $V_s$ = Cost obtained by operating the plant at the model's optimal 
	input and $V^{*}_s$ = plant optimal cost.
\end{itemize}

\end{frame}

\begin{frame}{Steady-state cost curves (Cost Type 2)}

	\begin{figure}
		\centering
		\includegraphics[page=9, height=0.8\textheight, 
		width=0.6\textwidth]{reac_plots.pdf}
	\end{figure}

\end{frame}

\begin{frame}{Optimization analysis (Cost Type 2)}

\begin{figure}
	\centering
	\includegraphics[page=16, height=0.7\textheight, 
	width=0.48\textwidth]{reac_plots.pdf}
	\includegraphics[page=17, height=0.7\textheight, 
	width=0.48\textwidth]{reac_plots.pdf}
\end{figure}


\begin{itemize}
	\item Generate 500 cost parameters in a range $\begin{bmatrix} 
		\underline{p}$, $\overline{p} \end{bmatrix}$ and examine the 
		steady-state optimization solutions of the plant and identified model.
	\item Optimization analysis is performed for the Hybrid model with all the 
	grey-box states.
\end{itemize}

\end{frame}

\section{Conclusions}
\begin{frame}{Conclusions}

	\begin{block}{}

	\begin{itemize}
		\item We can \alert{optimize over the hybrid process models obtained 
		with neural networks} if we are careful about the function 
		parameterization choices with the networks.
		\medskip
		\item Do 1-2 more hybrid modeling examples, with an emphasis on process 
		optimization.
	\end{itemize}

	\end{block}

\end{frame}

\begin{frame}{References}
\bibliographystyle{abbrvnat}
\bibliography{articles,proceedings,books,unpub, resgrppub}
\end{frame}

\end{document}