\documentclass[preprint,5p, twocolumn, authoryear]{elsarticle}

%% Use the option review to obtain double line spacing %
%\documentclass[preprint,review,12pt]{elsarticle}

%% Use the options 1p,twocolumn; 3p; 3p,twocolumn; 5p; or 5p,twocolumn % for a
%journal layout: % \documentclass[final,1p,times]{elsarticle} %
%\documentclass[final,1p,times,twocolumn]{elsarticle} %
%\documentclass[final,3p,times]{elsarticle} %
%\documentclass[final,3p,times,twocolumn]{elsarticle} %
%\documentclass[final,5p,times]{elsarticle} %
%\documentclass[final,5p,times,twocolumn]{elsarticle}


%% The graphicx package provides the includegraphics command.
\usepackage[dvipsnames]{xcolor}
%\usepackage{color} \usepackage{colortbl} \usepackage{xcolor}
\usepackage{graphicx}
\usepackage{makecell}
%% The amssymb package provides various useful mathematical symbols
\usepackage{amsmath}   % Extra math commands and environments from the AMS
\usepackage{amssymb}   % Special symbols from the AMS
\usepackage{amsthm}    % Enhanced theorem and proof environments from the AMS
%\usepackage{latexsym}  % A few extra LaTeX symbols \usepackage{array}
%\usepackage{makecell} \usepackage{units} \usepackage{xcolor} % The lineno
%packages adds line numbers. Start line numbering with % \begin{linenumbers},
%end it with \end{linenumbers}. Or switch it on % for the whole article with
%\linenumbers after \end{frontmatter}. \usepackage{lineno}
%\usepackage[singlelinecheck=false]{caption}  

\RequirePackage{booktabs} \RequirePackage{mpcsymbols} \RequirePackage{units_jbr}
\usepackage{tikz, tikzsettings}
\usetikzlibrary{shapes,calc,spy, calc, backgrounds,arrows, fit,
  decorations.pathmorphing, decorations.pathreplacing, matrix, positioning} 
  
%\biboptions{longnamesfirst, semicolon} % natbib.sty is loaded by default.
%However, natbib options can be % provided with \biboptions{...} command.
%Following options are % valid:

%%   round  -  round parentheses are used (default) %   square -  square
%brackets are used   [option] %   curly  -  curly braces are used      {option}
%%   angle  -  angle brackets are used    <option> %   semicolon  -  multiple
%citations separated by semi-colon %   colon  - same as semicolon, an earlier
%confusion %   comma  -  separated by comma %   numbers-  selects numerical
%citations %   super  -  numerical citations as superscripts %   sort   -  sorts
%multiple citations according to order in ref. list %   sort&compress   -  like
%sort, but also compresses numerical citations %   compress - compresses without
%sorting
%%
%% \biboptions{comma,round}

% \biboptions{}

\newcommand{\underl}[1]{\underline{#1}} \newcommand{\overl}[1]{\overline{#1}}
\newcommand{\satlb}[1]{\textnormal{sat}(#1, \underline{#1}, \overline{#1})}
\newcommand{\tif}{\textnormal{if}}

\newcommand{\degC}{\ensuremath{^\circ}{\rm C}} \newcommand{\oT}{\overline{T}}
\newcommand{\uT}{\underline{T}}


\graphicspath{{./}{./figures/}{./figures/paper/}}

%\newdefinition{assumption}{Assumption} \newdefinition{prop}{Proposition}
%\newdefinition{definition}{Definition} \newdefinition{rmk}{Remark}
%\newtheorem{thm}{Theorem}

\journal{Computers and Chemical Engineering}

\begin{document}

\begin{frontmatter}

%% Title, authors and addresses

\title{Hybrid process modeling strategies for real-time optimization and control}

%% use the tnoteref command within \title for footnotes; % use the tnotetext
%command for the associated footnote; % use the fnref command within \author or
%\address for footnotes; % use the fntext command for the associated footnote; %
%use the corref command within \author for corresponding author footnotes; % use
%the cortext command for the associated footnote; % use the ead command for the
%email address, % and the form \ead[url] for the home page:
%%
%% \title{Title\tnoteref{label1}} % \tnotetext[label1]{}
\author[label1]{Pratyush Kumar\corref{cor1}}
\ead{pratyushkumar@ucsb.edu}
%% \ead[url]{home page} % \fntext[label2]{}
\cortext[cor1]{Corresponding Author}
%% \fntext[label3]{}

\author[label1]{James B. Rawlings}
\ead{jbraw@ucsb.edu}

\address[label1]{Department of Chemical Engineering, University of California,
        Santa Barbara, CA 93106, United States} 
                
\begin{abstract}

In industrial applications, model predictive control (MPC) is installed in a
hierarchical operational framework above regulatory layer proportional-integral
(PI) controllers. Traditional MPC implementations do not include closed-loop
dynamics of feedback loops regulated by the PI controllers in MPC optimization
problems. This paper presents case studies to demonstrate the advantages of
modeling the closed-loop dynamics of PI feedback loops in MPC optimization
formulations. We show that this modeling step has two key advantages, (i)
systematic treatment of unmeasured disturbances in the PI feedback loops, (ii)
improving closed-loop transient dynamics of cascade control architectures for
slow PI feedback loops. Case studies on a continuous stirred tank reactor (CSTR)
example and an heating, ventilation, and air-conditioning (HVAC) application are
presented to highlight both these advantages of modeling PI controllers in MPC
formulations. 

\end{abstract}

\begin{keyword}
Cascade control \sep model predictive control \sep proportional-integral control
\sep economic MPC
\end{keyword}

\end{frontmatter}

%% main text
\section{Introduction} \label{sec:introduction}


\section*{Acknowledgment}
The funding for this work was provided by the industrial members of the
Texas-Wisconsin-California Control Consortium (TWCCC). 

%% New version of the num-names style
\bibliographystyle{elsarticle-num-names}
\bibliography{abbreviations,articles,books,unpub,proceedings}

\end{document}
